\documentclass{article}
\usepackage{fullpage}
\usepackage[utf8]{inputenc}
\usepackage[pdfpagelabels=true,linktocpage]{hyperref}
\usepackage{color}

\title{\texttt{SMT-RAT 24.06}}

\begin{document}

\maketitle

\texttt{SMT-RAT}~\cite{Corzilius2015} is an open-source \texttt{C++} toolbox for strategic and parallel SMT solving consisting of a collection of SMT compliant implementations of methods for
solving quantifier-free first-order formulas with a focus on non-linear real and integer arithmetic.
Further supported theories include linear real and integer arithmetic, difference logic, bit-vectors and pseudo-Boolean constraints.
A more detailed description of \texttt{SMT-RAT} can be found at \href{https://smtrat.github.io/}{\color{blue}https://smtrat.github.io/}.


For \emph{quantifier-free non-linear real arithmetic (QF\_NRA)}, \texttt{SMT-RAT} uses our implementation of the MCSAT framework \cite{Moura2013} inspired by \cite{jovanovic2012solving}.
We employ incomplete methods to handle simpler problem classes more efficiently.
Thus, our implementation is equipped with multiple explanation backends based on Fourier-Motzkin variable elimination, interval constraint propagation, virtual substitution as in \cite{Abraham2017}, and a novel level-wise variant of the one-cell CAD \cite{brown2015constructing,nalbach2024}, which are called in this order.
The level-wise one-cell CAD uses linear approximations of some cell boundaries which would otherwise be defined by polynomials with high degree, as described in \cite{promies_msc}.
The general MCSAT framework is integrated in our adapted \texttt{minisat}~\cite{Een2003} solver.
Our variable ordering is fully dynamic as suggested in \cite{Jovanovic2013}.
Furthermore, we supplement our solver with an incomplete check for subtropical satisfiability~\cite{Fontaine2017} before the main MCSAT solver is called.
For algebraic operations, we use libpoly~\cite{libpoly}.

For \emph{non-linear real arithmetic (NRA)}, \texttt{SMT-RAT} uses the cylindrical algebraic covering (CAlC) method \cite{Abraham2020} extended for quantifiers \cite{kremer2023}.

%\newpage

\paragraph{Current authors}
Jasper Nalbach, Valentin Promies, Erika \'Abrah\'am
(Theory of Hybrid Systems Group, RWTH Aachen University).

\paragraph{Previous contributions by former group members}
Gereon Kremer (currently at Certora),
Florian Corzilius,
Rebecca Haehn,
Sebastian Junges,
Stefan Schupp.


\bibliographystyle{plain}
\bibliography{literature}

\end{document}
